\documentclass{article}

\begin{document}
\title{TikzEdt Doku und TODO-Liste}
\date{}
\maketitle


   \section{MainWindow und UI}

      {\bf Aufgaben der Komponente:}
\begin{itemize}
    \item Integriert alle anderen Teile des Programms.
      \item  Sieht gut aus.
\end{itemize}

      {\bf  ToDo:}
\begin{itemize}
       \item  sehr vieles,.... muss aber z.T. noch warten, bis die anderen Komponenten funkionieren.
        \item  Gutes Icon fuer TikzEdt erfinden.
\end{itemize}

\subsection{Der Code-Editor auf der Hauptseite  (avalonedit) }
      {\bf ToDo:}
\begin{itemize}
        \item  Syntax-Highlighting konfigurieren.
            Konkret: xml-Datei TikzSyntax.xshd vervollstaendigen.
       \item (Evtl.) Auto-Vervollstaendigung implementieren. (Komplizierter, benoetigt je nach Perfektionsanspruch den Parser)
\end{itemize}


   \section{Klasse TikzDisplay (Kompilierung und Anzeige des Bildes)}

     {\bf Aufgaben der Komponente:}
\begin{itemize}
    \item Nimmt einen string "code", kompiliert den und zeigt ihn an. (Methode Compile)
  \item Erzeugt events (event OnCompileEvent) um den Status anzuzeigen.
\end{itemize}

Die Kompilierung und Anzeige funktioniert groessenteils. Details muessen noch verbessert werden.

      {\bf  ToDo:}
\begin{itemize}
       \item Fehlermeldungen von pdflatex abfangen und mit ausgeben. (Am besten mi Zeilennummer, Spaltennummer)
        \item Manchmal blockiert pdflatex bei einem Fehler anstatt mit Fehlermeldung abzubrechen.  Loesen oder im schlimmsten Fall den Prozess nach Timeout killen. (Anmerkung: das gleiche Phaenomen gibt es bei ktikz.)
\end{itemize}

  \section{Klasse TikzParser (Parsen von TikzCode)}

     {\bf Aufgaben der Komponente:}
\begin{itemize}
    \item Hat nur eine relevante Methode:\begin{verbatim} Parse(string code)\end{verbatim} Parsed den code und erzeugt einen Parsetree (Klasse  \begin{verbatim}Tikz_ParseTree\end{verbatim}) oder null im Fehlerfall. (Anmerkung: Ist eine "best effort" Komponente. Es wird nicht jede Tikz-Konstruktion erkannt.)
\end{itemize}

Ich habe eine Antlr-Grammatik geschrieben, die einfache Tikz-Files parsen kann.

      {\bf  ToDo:}
\begin{itemize}
       \item  Man muss den parse-tree den antlr liefert noch zu einem  Tikz$\_$ParseTree verarbeiten.
        \item Man kann beliebig viel Zeit investieren, Tikz noch genauer abzubilden.
\end{itemize}

\subsection{Klasse Tikz$\_$ParseTree }
     {\bf Aufgaben der Komponente:}
\begin{itemize}
    \item Enthaelt den ParseTree.
\end{itemize}

      {\bf  ToDo:}
\begin{itemize}
       \item Muss groesstenteils noch geschrieben werden. Es steht nur ein Geruest.
        \item Insbesondere fehlt eine Methode GetBB() die ein estimate der BoundingBox liefert. (=maximum der vorkommenden Koordinaten)
\end{itemize}

\section{Klasse PdfOverlay}
 
    {\bf Aufgaben der Komponente:}
\begin{itemize}
    \item Halbtransparentes Control, das ueber dem Pdf-output angezeigt wird. 
\item Nimmt einen \begin{verbatim}Tikz_ParseTree\end{verbatim} und zeigt die enthaltenen Knoten (und Kanten?) an.
\item Enthaelt Editierfunktionen: (i) Dinge auswaehlen und verschieben (ii) Nodes hinzufuegen (iii) Kanten hinzufuegen  .... (iv) +X
\item Bei Aenderung des ParseTrees wird ein Event ausgeloest. (Die Mainwindow muss daraufhin den code im editor aktualisieren)
\end{itemize}

Bisher zeigt die Klasse ein paar Knoten an, die man verschieben kann, aber es wird noch kein Code editiert. (Ist also ein Mockup)

      {\bf  ToDo:}
\begin{itemize}
       \item Muss groesstenteils noch geschrieben werden. 
\end{itemize}

\section{SnippetManager????}
Man sollte dem vergesslichen Nutzer (wie mir) noch helfen, sich an Befehle etc zu erinnern.
Ich weiss noch nicht, wie genau man das anstellt. Ein Vorschlag: Ein Snippetmanager.
Enthalt eine Bibliothek von Snippets, die per Menu oder Toolbar oder PopupFenster etc. abgerufen werden koennen und an die Cursorposition eingefuegt werden.

Einige Ideen (Brainstorming)
\begin{itemize}
\item Jedes Snippet enthselt (i) den SnippetText (klar) (ii) es hat eine Kategorie, (z.B. Template fuer ganze Vorlagen, Edge/Vertex-Style fuer Stye-Befehle, etc...) und (iii) ein einfaches komplettes tikzpicture, das die Benutzung einfachstmoeglich demonstriert. Das Picture kann man rendern und neben dem Snippet als Bild anzeigen.
\item Der Benutzer kann Snippets hinzufuegen. Das ersetzt insbesondere den ``Stylemanager'' aus GraphToTikz, und ausserdem eine Vorlegenverwaltung.
\end{itemize}

      {\bf  ToDo:}
\begin{itemize}
       \item Muss noch entworfen und geschrieben werden.
\item Was ist die beste Realisierung? Ist es noetig?
\end{itemize}

\end{document}