\documentclass[]{article}

\usepackage{layouts}
\usepackage{tikz}

\usepackage[active,tightpage]{preview}
\PreviewEnvironment{tikzpicture}

\usetikzlibrary{calc}

\begin{document}

\tikzstyle{every node}= [font=\Huge]



\begin{tikzpicture}
	
	\node at (0,3) {textwidth in cm: \printinunitsof{cm}\prntlen{\textwidth}};
	

\node(v1) at (7,0) {?};
\node(v2) at (7,1) {?};

\path
  let
    \p1=(v1),
    \p2=(v2)
  in {
  node at (7,-1) { \x1,  \y1}
  node at (7,-2) { \x2,  \y2}
  };


\node(v3) at (-2,-1) {$\frac{\\x+\\y}{2}=$};

\draw
  let
    \p1=(v3)
  in
  \pgfextra
  {
  	\pgfmathparse{0.5* ( \y1/28.4527 + \x1/28.4527 )}
  	\pgfmathresult pt
  }
    (1,-1) 
    node[right] {  \pgfmathresult pt };
    
    \draw
  let
    \p1=(v1)
  in
  \pgfextra
  {
  	\pgfmathparse{0.5* ( \y1/28.4527 + \x1/28.4527 )}
  	
  }
    (-1,\pgfmathresult pt) 
    node[right] {  \pgfmathresult pt };
    
    
   \foreach \x in {5,10,...,35}
    \draw (-9 cm,\x mm) -- (-6 cm,\x mm) 
    node at (-6.5 cm,\x mm) {\pgfmathparse{4*\x}\pgfmathprintnumber{\pgfmathresult}};


\node(v11) at (3,-7) {};
\path
  let
    \p1=(v11)  
  in {
  node at (v11)
	  {
	  	\pgfmathparse{(\x1/28.4527)}
	  	\global\let\xpos\pgfmathresult
	  	\pgfmathparse{(\y1/28.4527)}
	  	\global\let\ypos\pgfmathresult
	  	This node is at \pgfmathprintnumber{\xpos}; \pgfmathprintnumber{\ypos}
	  }
  };

  
\node at (-5,-5) {?};
\end{tikzpicture}

\end{document}