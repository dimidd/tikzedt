\documentclass[]{article}

\usepackage{layouts}
\usepackage{tikz}



%for environment aligned
\usepackage{amsmath}
\usepackage{ifthen}
\usepackage[active,tightpage]{preview}
\PreviewEnvironment{tikzpicture}

\usetikzlibrary{calc}
\usetikzlibrary{arrows}

\begin{document}

\newcommand{\degree}{\ensuremath{^\circ}}

%note: align= is needed to print nodes with several lines
\tikzstyle{every node}= [font=\Huge, align=center]
%\tikzstyle{every node}= [font=\Huge, draw=black,                     minimum width=2cm, thick]


 \tikzstyle{green_arrow_edge} = [->, draw=green!55, line width=5]


\begin{tikzpicture}
	
\pgfmathparse{28.4527}
\global\let\tikzconst\pgfmathresult
	
\clip (-24,4) rectangle (-5,-15);
	
\pgfmathparse{-8.0}
\global\let\nooverlay\pgfmathresult
\node (c1) at (-16,\nooverlay) {c1};
\node (c2) at (-18,-3) {c2};
\node (c3) at (-10,\nooverlay) {c3};

\path
  let
    \p1=(c1)  
  in {
  node
	  {
	  	\pgfmathparse{(\x1/\tikzconst)}
	  	\global\let\conex\pgfmathresult
	  	\pgfmathparse{(\y1/\tikzconst)}
	  	\global\let\coney\pgfmathresult
	  }
  };
\path
  let
    \p1=(c2)  
  in {
  node {
	  	\pgfmathparse{(\x1/\tikzconst)}
	  	\global\let\ctwox\pgfmathresult
	  	\pgfmathparse{(\y1/\tikzconst)}
	  	\global\let\ctwoy\pgfmathresult
	  }
  };
  \path
  let
    \p1=(c3)  
  in {
  node
	  {
	  	\pgfmathparse{(\x1/\tikzconst)}
	  	\global\let\cthreex\pgfmathresult
	  	\pgfmathparse{(\y1/\tikzconst)}
	  	\global\let\cthreey\pgfmathresult
	  }
  };
  

  \node
	  {
	  	%calculate distances c1 - c3
	  	\pgfmathparse{(\cthreex - \conex)}
	  	\global\let\distxonethree\pgfmathresult
	  	\pgfmathparse{(\cthreey - \coney)}
	  	\global\let\distyonethree\pgfmathresult
	  	%calculate start angle c1 - c3 -perpendicular
	  	\pgfmathparse{acos(\distxonethree / sqrt(\distxonethree*\distxonethree+\distyonethree*\distyonethree) )}
	  	\global\let\startangle\pgfmathresult	 	  	
	  	%calc start direction
	  	\pgfmathparse{\conex*\cthreey < \coney*\cthreex ? 90-\startangle : \startangle}
	  	\global\let\startangle\pgfmathresult	 
	  	%calculate angle c3 - c1 - c2
	  	%helper variables
	  	\pgfmathparse{sqrt(pow(\conex - \ctwox,2) + pow(\coney - \ctwoy,2))}
	  	\let\varb\pgfmathresult
	  	\pgfmathparse{sqrt(pow(\cthreex - \conex,2) + pow(\cthreey - \coney,2))}
	  	\let\vara\pgfmathresult
	  	\pgfmathparse{sqrt(pow(\cthreex - \ctwox,2) + pow(\cthreey - \ctwoy,2))}
	  	\let\varc\pgfmathresult
	  	%%now calculate the angle
	  	\pgfmathparse{((\vara*\vara) + (\varb*\varb) - (\varc*\varc)) / (2 * \vara * \varb)}
	  	\pgfmathparse{acos(\pgfmathresult)}
	  	\global\let\angle\pgfmathresult	  	
	  	%%	check direction of given three coordiantes
	  	\pgfmathparse{\conex*\ctwoy < \coney*\ctwox ? \angle : 360-\angle}
	  	\global\let\angle\pgfmathresult						
	  };

 



\draw[green_arrow_edge] (c1) edge (c3);
\draw[green_arrow_edge] (c1) edge (c2);

%\draw[green_arrow_edge, bend right=35] (c3) edge (c2);
\draw[green_arrow_edge] (c3) arc (\startangle:\angle:\distxonethree);

\node at (-14,-5) {\angle \degree};
\node at (-14,1) {Move node c2 but not too far \\ (math behind is not perfect)};

\end{tikzpicture}

\end{document}